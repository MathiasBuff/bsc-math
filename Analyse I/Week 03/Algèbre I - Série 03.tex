% !TeX spellcheck = fr_FR

% TODO: Replace scan images with clean text where possible

\documentclass[a4paper, 10pt]{report}

\usepackage[french]{babel}
\usepackage[T1]{fontenc}

\usepackage{amsmath, amssymb, amsfonts}

\usepackage{hyperref}
\usepackage{geometry}

\usepackage{xcolor}
\usepackage{graphicx}

\usepackage{fancyhdr}
\usepackage{lastpage}

\usepackage{enumitem}

\geometry{
	a4paper,
	left=25mm,
	right=25mm,
	top=35mm,
	bottom=25mm,
	headsep=5mm,
	headheight=20mm,
}

\definecolor{solution}{HTML}{E5E4E2}
\providecommand{\abs}[1]{\lvert#1\rvert}
\providecommand{\norm}[1]{\lVert#1\rVert}
\DeclareMathOperator{\card}{card}

\begin{document}
	
	\renewcommand{\headrule}{%
		\vspace{-4pt}\hrulefill
		\raisebox{-6.8pt}{\ \includegraphics[height=5mm]{../../icon.png}}
		\hrulefill
	}	
	\pagestyle{fancy}
	\fancyhf{}
	
	\fancyhead[L]{\small \slshape Automne 2024}
	\fancyhead[C]{\Large \bfseries Analyse I - Série 03}
	\fancyhead[R]{\small Buff Mathias}
	\fancyfoot[L]{
		\small Source files available at:
		\href{https://github.com/MathiasBuff/bsc-math}
		{github.com/MathiasBuff/bsc-math}
	}
	\fancyfoot[R]{
		\small Page \thepage
		\hspace{1pt} /
		\pageref*{LastPage}
	}
	

	\noindent
	\textbf{Exercice 1.} Soient $E, f \subset \mathbb{R}$ et
	$f : e \to F$ une fonction bijective et monotone. Est-ce que
	$f^{-1}$ est monotone ?
	
	\vspace{5mm}
	\noindent
	\textbf{Exercice 2.} Les fonctions suivantes sont-elles bien
	définies ? injectives ? surjectives ? bijectives ?
	
	\begin{enumerate}[label=(\roman*)]
		\item $\begin{aligned}
			a :\ & \mathbb{N} &&\to &&\mathbb{N}\\
			&n &&\mapsto &&n + 1
		\end{aligned}$
		%
		\vspace{5pt}
		%
		\item $\begin{aligned}
			b :\ & \mathbb{R} &&\to &&\mathbb{R}\\
			&x &&\mapsto &&2x
		\end{aligned}$
		%
		\vspace{5pt}
		%
		\item $\begin{aligned}
			c :\ & [0, \infty) &&\to &&(-\infty, 0]\\
			&x &&\mapsto &&x^2
		\end{aligned}$
		%
		\vspace{5pt}
		%
		\item $\begin{aligned}
			d :\ & \mathbb{N} &&\to &&\{-1, 1\}\\
		&n &&\mapsto &&\left\{\begin{aligned}
			&1 & &\text{si}\ n\ \text{est pair},\\
			-&1 & &\text{si}\ n\ \text{est impair}
		\end{aligned}\right.\end{aligned}$
		%
		\vspace{5pt}
		%
		\item $\begin{aligned}
			e :\ & \mathbb{N} &&\to &&\{-1, 1\}\\
			&n &&\mapsto &&(-1)^n
		\end{aligned}$
	\end{enumerate}
	
	\vspace{5mm}
	\noindent
	\textbf{Exercice 3.} Soit $E \subset \mathbb{R}$. Montrer que
	$\sup{E}$, s'il existe, est unique.
	
	\vspace{5mm}
	\noindent
	\textbf{Exercice 4.} Trouver le supremum et infimum dans $\mathbb{R}$
	de :
	\begin{enumerate}[label=(\roman*)]
		\item $E = \{\frac{1}{n} + (-1)^n : n \in \mathbb{N}^*\}$
		\item $E = \{x \in \mathbb{R} : 0 \leq x < 1\}$
		\item $E = \{x \in \mathbb{R} : -8 \leq x^3 \leq -1\
			\text{ou}\ 2 \leq x + 1 < 6\}$	
	\end{enumerate}
	Est-ce que ce sont des maximums et des minimums ?
	
	\vspace{5mm}
	\noindent
	\textbf{Exercice 5.} Déterminer quelles sont les fonctions
	injectives, surjectives, et bijections parmi la liste suivante.
	Justifier vos affirmations.
	\begin{enumerate}[label=(\roman*)]
		\item $f : \mathbb{R} \setminus \{0\} \to
			\mathbb{R} \setminus \{0\}\\
		\phantom{f :\ } x \mapsto \frac{1}{x}$
		%
		\item $g : \mathbb{N} \setminus \{0, 1\} \to \mathbb{N}\\
		\phantom{g :\ } n \mapsto \text{le plus petit nombre premier
			divisant}\ n$
		%
		\item Soit $E$ un ensemble,
		\[\begin{aligned}
			\chi :\ & \mathcal{P}(E) &\to \quad &\{0, 1\}^E\\
			&A &\mapsto \quad &\chi_A
		\end{aligned}\]
		où $\chi_A$ est la fonction caractéristique de l'ensemble $A$.
	\end{enumerate}
	
	\vspace{5mm}
	\noindent
	\textbf{Exercice 6.} Déterminer quelles sont les fonctions
	croissantes et décroissantes parmi la liste suivante.
	Justifier vos affirmations.
	\begin{enumerate}[label=(\roman*)]
		\item $\begin{aligned}
			h :\ & \mathbb{R} &&\to &&\mathbb{R}\\
			&x &&\mapsto &&x^2
		\end{aligned} \qquad$ même question avec
		$\qquad \begin{aligned}
			i :\ & [0, \infty) &&\to &&\mathbb{R}\\
			&x &&\mapsto &&x^2
		\end{aligned}$
		%
		\vspace{10pt}
		%
		\item $\begin{aligned}
			\pi :\ & \mathbb{N} &&\to &&\mathbb{N}\\
			&n &&\mapsto &&\pi(n)
		\end{aligned} \qquad$ où $\pi(n)$ est le nombre de nombres
		premiers inférieurs ou égaux à $n$.
	\end{enumerate}
	
	\vspace{5mm}
	\noindent
	\textbf{Exercice 7.} Soient $E \subset F \subset \mathbb{R}$.
	Montrer que $\sup{E} \leq \sup{F}$ et $\inf{E} \geq \inf{F}$.
		
	\vspace{5mm}
	\noindent
	\textbf{Exercice 8.} Soit $f : E \to F$. Montrer que
	\[
		E = \bigcup\limits_{y \in F} f^{-1}(\{y\})
	\]
	
	\vspace{5mm}
	\noindent
	\textbf{Exercice 9.} Soient $f : E \to F$ et $g : F \to G$ deux
	fonctions.
	\begin{enumerate}[label=(\roman*)]
		\item Supposons que $g \circ f$ est injective, est-ce que
		$f$ est injective ? Même question avec $g$ ?
		%
		\item Supposons que $g \circ f$ est surjective, est-ce que
		$f$ est surjective ? Même question avec $g$ ?
		%
		\item Est-ce que $g \circ f$ bijective implique $f$ et $g$
		bijectives ?
	\end{enumerate}	
	Pour chaque question, si la réponse est oui, le prouver.
	Sinon, exhiber un contre-exemple.
	
	\fancyhf{}
	\renewcommand{\headrule}
	{\rule{\textwidth}{0pt}}
	\fancyfoot[R]{
		\small Page \thepage
		\hspace{1pt} /
		\pageref*{LastPage}
	}
	
\end{document}
