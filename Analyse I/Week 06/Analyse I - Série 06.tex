% !TeX spellcheck = fr_FR

% TODO: Replace scan images with clean text where possible

\documentclass[a4paper, 10pt]{report}

\usepackage[french]{babel}
\usepackage[T1]{fontenc}

\usepackage{amsmath, amssymb, amsfonts}

\usepackage{hyperref}
\usepackage{geometry}

\usepackage{xcolor}
\usepackage{graphicx}

\usepackage{fancyhdr}
\usepackage{lastpage}

\usepackage{enumitem}

\geometry{
	a4paper,
	left=25mm,
	right=25mm,
	top=35mm,
	bottom=25mm,
	headsep=5mm,
	headheight=20mm,
}

\definecolor{solution}{HTML}{E5E4E2}
\providecommand{\abs}[1]{\lvert#1\rvert}
\providecommand{\norm}[1]{\lVert#1\rVert}
\DeclareMathOperator{\card}{card}

\begin{document}
	
	\renewcommand{\headrule}{%
		\vspace{-4pt}\hrulefill
		\raisebox{-6.8pt}{\ \includegraphics[height=5mm]{../../icon.png}}
		\hrulefill
	}	
	\pagestyle{fancy}
	\fancyhf{}
	
	\fancyhead[L]{\small \slshape Automne 2024}
	\fancyhead[C]{\Large \bfseries Analyse I - Série 06}
	\fancyhead[R]{\small Buff Mathias}
	\fancyfoot[L]{
		\small Source files available at:
		\href{https://github.com/MathiasBuff/bsc-math}
		{github.com/MathiasBuff/bsc-math}
	}
	\fancyfoot[R]{
		\small Page \thepage
		\hspace{1pt} /
		\pageref*{LastPage}
	}
	

	\noindent
	\textbf{Exercice 1.} Soit $E \subset \mathbb{R}$ et définissons
	\[\overline{E}:= \{a \in \mathbb{R} : \exists (u_n) \in E^{\mathbb{N}}\
		\text{convergeant vers}\ a\}\]
	\begin{enumerate}[label=(\roman*)]
		\item Montrer que $E \subset \overline{E}$
		\item Montrer que $\overline{(0, 1)} = \overline{[0, 1]} = [0, 1]$
		\item Montrer que $\overline{E}$ est l'ensemble des points de
		$\mathbb{R}$ qui se situent à une distance arbitrairement proche
		d'un point de $E$, c'est-à-dire
		\[\overline{E}= \{a \in \mathbb{R} : \forall \varepsilon > 0,
			\exists x \in E, \abs{a - x} < \varepsilon\}\]
	\end{enumerate}
	
	\vspace{5mm}
	\noindent
	\textbf{Exercice 2.} Étudier les points de continuité des fonctions
	suivantes.
	\begin{enumerate}[label=(\roman*)]
		\item La fonction $f(x) = \frac{x-5}{x+3}$ de
		$\mathbb{R}\setminus\{-3\}$ dans $\mathbb{R}$
		%
		\item La fonction $f(x) = x\chi_{\mathbb{Q}}(x)$ de
		$\mathbb{R}$ dans $\mathbb{R}$
		%
		\item La fonction $f(x) =
			\left\{\begin{aligned}
				&\tfrac{x^2 + 2\abs{x}}{x} &\text{si}\ x \neq 0\\
				&3						  &\text{si}\ x  =   0
			\end{aligned}\right.$ de $\mathbb{R}$ dans $\mathbb{R}$
			%
		\item Soit $n \in \mathbb{N}^*$ impair. La fonction $f(x) =
			\left\{\begin{aligned}
				&\tfrac{x-1}{x^n-1} &\text{si}\ x \neq 1\\
				&\tfrac{1}{n}		&\text{si}\ x  =   1
			\end{aligned}\right.$ de $\mathbb{R}$ dans $\mathbb{R}$
	\end{enumerate}
	
	\vspace{5mm}
	\noindent
	\textbf{Exercice 3.} (Théorème des suites alternées)\\
	Soit $(u_n)$ une suite décroissante qui converge ver 0, et posons
	$\displaystyle S_n = \sum_{k=0}^{n}(-1)^ku_k$.
	\begin{enumerate}[label=(\roman*)]
		\item Montrer que $(S_{2n})$ est décroissante et  $(S_{2n+1})$
		croissante. En déduire que  $(S_n)$ converge.
		%
		\item Montrer que pour tout $m \geq n, \abs{S_n - S_m} \leq u_{n+1}$
		(\textit{Indication :} utiliser le fait que $u_k - u_{k+1} \geq 0$).
		En déduire que si $S = \lim S_n$, alors $\abs{S_n - S} \leq u_{n+1}$.
		%
		\item Montrer que la suite $\displaystyle
			\sum_{k=1}^{n}\frac{(-1)^{k+1}}{k}$ converge.\\
		(Vous montrerez durant le second semestre que la limite vaut
		$\log2$.)
	\end{enumerate}
	
	
	\vspace{5mm}
	\noindent
	\textbf{Exercice 4.} Soit $g: [0, \infty) \to [0, \infty)$ une
	fonction continue en 0 telle que $g(0) = 0$.\\
	Soient $f: E \to \mathbb{R}$ et $a \in E$.
	Montrer que $f$ est continue en $a$ si et seulement si
	\[
		\forall \varepsilon > 0, \exists \delta > 0, \forall x \in E,
		\abs{x - a} < \delta \implies \abs{f(x) - f(a)} \leq g(\varepsilon)
	\]
		
	\vspace{5mm}
	\noindent
	\textbf{Exercice 5.} Soit $f : \mathbb{R} \to \mathbb{R}$
	\begin{enumerate}[label=(\roman*)]
		\item Montrer que si $f$ est continue sur $\mathbb{R}$, alors
		$\abs{f}$ est continue.
		%
		\item Donner un exemple de fonction $f$ telle que $\abs{f}$ est
		continue mais $f$ est discontinue en tout point. 
		%
		\item Exprimer $\max(x, y)$ et $\min(x, y)$ en utilisant la
		valeur absolue.
		%
		\item Montrer que si $f$ et $g$ sont continues sur $\mathbb{R}$,
		alors $\max(f, g)$ et $\min(f, g)$ sont également continues.
	\end{enumerate}
	
	\newpage
	
	\vspace{5mm}
	\noindent
	\textbf{Exercice 6.} Soit $P$ un polynôme à coefficients réels de
	degré impair. Montrer qu'il existe une racine $c \in \mathbb{R}$
	telle que $P(c) = 0$ (\textit{Indication :} on peut utiliser le
	théorème des valeurs intermédiaires). Pour tout $d \in \mathbb{N}$,
	donner un exemple de polynôme de degré $2d$ qui n'admet pas de
	racine sur $\mathbb{R}$.
	
	\vspace{5mm}	
	\noindent
	\textbf{Exercice 7.} (Théorème du point fixe)\\
	Soit $f: [0, 1] \to [0, 1]$ continue. Montrer qu'il existe
	$c \in [0, 1]$ tel que $f(c) = c$.\\
	\textit{Indication :} utiliser le théorème des valeurs intermédiaires.	
	
	\fancyhf{}
	\renewcommand{\headrule}
	{\rule{\textwidth}{0pt}}
	\fancyfoot[R]{
		\small Page \thepage
		\hspace{1pt} /
		\pageref*{LastPage}
	}
	
%	
%	
%	\colorbox{solution}
%	{
%		\begin{minipage}{0.9\textwidth}
%			s
%		\end{minipage}
%	}
%	
%	\colorbox{solution}
%	{
%		\begin{minipage}{0.9\textwidth}
%			\begin{enumerate}[label=(\alph*)]
%				\item a
%			\end{enumerate}
%		\end{minipage}
%	}
	
\end{document}
