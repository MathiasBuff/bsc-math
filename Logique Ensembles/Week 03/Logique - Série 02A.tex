% !TeX spellcheck = fr_FR

% TODO: Replace scan images with clean text where possible

\documentclass[a4paper, 10pt]{report}

\usepackage[french]{babel}
\usepackage[T1]{fontenc}

\usepackage{amsmath, amssymb, amsfonts}

\usepackage{hyperref}
\usepackage{geometry}

\usepackage{xcolor}
\usepackage{graphicx}

\usepackage{fancyhdr}
\usepackage{lastpage}

\usepackage{enumitem}

\geometry{
	a4paper,
	left=25mm,
	right=25mm,
	top=35mm,
	bottom=25mm,
	headsep=5mm,
	headheight=20mm,
}

\definecolor{solution}{HTML}{E5E4E2}
\providecommand{\abs}[1]{\lvert#1\rvert}
\providecommand{\norm}[1]{\lVert#1\rVert}

\begin{document}
	
	\renewcommand{\headrule}{%
		\vspace{-4pt}\hrulefill
		\raisebox{-6.8pt}{\ \includegraphics[height=5mm]{../../icon.png}}
		\hrulefill
	}	
	\pagestyle{fancy}
	\fancyhf{}
	
	\fancyhead[L]{\small \slshape Automne 2024}
	\fancyhead[C]{\Large \bfseries Logique et Théorie des Ensembles\\
		Série 02-A}
	\fancyhead[R]{\small Buff Mathias}
	\fancyfoot[L]{
		\small Source files available at:
		\href{https://github.com/MathiasBuff/bsc-math}
		{github.com/MathiasBuff/bsc-math}
	}
	\fancyfoot[R]{
		\small Page \thepage
		\hspace{1pt} /
		\pageref*{LastPage}
	}
	
	\noindent
	\textbf{Exercice 1.} Deux ensembles disjoints sont-ils
	nécessairement distincts ?	
	
	\vspace{5mm}
	\noindent
	\textbf{Exercice 2.} Montrer que
	
	\begin{itemize}
		\item $A \times (B \cup C) = (A \times B) \cup (A \times C)$,
		%
		\item $A \times (B \cap C) = (A \times B) \cap (A \times C)$.
	\end{itemize}
	
	\vspace{5mm}
	\noindent
	\textbf{Exercice 3.} Soient $A$ et $B$ deux ensembles. Montrer que
	$A \subset B \iff A \setminus B = \emptyset$
		
	\vspace{5mm}
	\noindent
	\textbf{Exercice 4.} (Formules d'associativité). Soit $E$ un
	ensemble, $(A_i)_{i \in I}$ une famille de $\mathcal{P}(E)$ et
	$(J_k)_{k \in K}$ une famille incluse dans $I$ et recouvrant $I$.
	Montrer que
	
	\[
		\bigcup\limits_{i \in I} A_i =
			\bigcup\limits_{k \in K}\bigcup\limits_{i \in J_k} A_i
		\quad \text{et} \quad
		\bigcap\limits_{i \in I} A_i =
			\bigcap\limits_{k \in K}\bigcap\limits_{i \in J_k} A_i
	\]
	
	\vspace{5mm}
	\noindent
	{\color{red}\textbf{Exercice 5.}}
	Soient $A$ et $B$ deux parties de $E$.\\
	Discuter et résoudre l'équation d'inconnue $X \subset E$ donnée par
	
	\[(A \cap X) \cup (B \cap X^c) = \emptyset\]
	
\end{document}
