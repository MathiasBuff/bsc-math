% !TeX spellcheck = fr_FR

% TODO: Replace scan images with clean text where possible

\documentclass[a4paper, 10pt]{report}

\usepackage[french]{babel}
\usepackage[T1]{fontenc}

\usepackage{amsmath, amssymb, amsfonts}

\usepackage{hyperref}
\usepackage{geometry}

\usepackage{xcolor}
\usepackage{graphicx}

\usepackage{fancyhdr}
\usepackage{lastpage}

\usepackage{enumitem}

\geometry{
	a4paper,
	left=25mm,
	right=25mm,
	top=35mm,
	bottom=25mm,
	headsep=5mm,
	headheight=20mm,
}

\definecolor{solution}{HTML}{E5E4E2}
\providecommand{\abs}[1]{\lvert#1\rvert}
\providecommand{\norm}[1]{\lVert#1\rVert}

\begin{document}
	
	\renewcommand{\headrule}{%
		\vspace{-4pt}\hrulefill
		\raisebox{-6.8pt}{\ \includegraphics[height=5mm]{../../icon.png}}
		\hrulefill
	}	
	\pagestyle{fancy}
	\fancyhf{}
	
	\fancyhead[L]{\small \slshape Automne 2024}
	\fancyhead[C]{\Large \bfseries Logique et Théorie des Ensembles\\
		Série 03-B}
	\fancyhead[R]{\small Buff Mathias}
	\fancyfoot[L]{
		\small Source files available at:
		\href{https://github.com/MathiasBuff/bsc-math}
		{github.com/MathiasBuff/bsc-math}
	}
	\fancyfoot[R]{
		\small Page \thepage
		\hspace{1pt} /
		\pageref*{LastPage}
	}
	
	\noindent
	\textbf{Exercice 1.} Montrer que $(A \setminus B) \setminus C
		= A \setminus (B \cup C)$.
	
	\vspace{5mm}
	\noindent
	\textbf{Exercice 2.}  Montrer que $(A \cap B) \times (C \cap D)
		= (A \times C) \cap (B \times D)$.
	
	\vspace{5mm}
	\noindent
	\textbf{Exercice 3.} (Lois de De Morgan généralisées).\\
	Soient $I \neq \emptyset$ un ensemble et $(A_i)_{i \in I}$
	une famille de parties de $E$ et $A \subset E$. Montrer que
	
	\[
		A \cap \left(\bigcup\limits_{i \in I} A_i\right) =
			\bigcup\limits_{i \in I}(A \cap A_i),
		\quad
		A \cup \left(\bigcap\limits_{i \in I} A_i\right) =
			\bigcap\limits_{i \in I}(A \cap A_i)
	\]
	\[
		A \setminus \bigcup\limits_{i \in I} A_i =
			\bigcap\limits_{i \in I} (A \setminus A_i),
		\quad
		A \setminus \bigcap\limits_{i \in I} A_i =
			\bigcup\limits_{i \in I} (A \setminus A_i)
	\]
		

	\vspace{5mm}
	\noindent
	{\color{red}\textbf{Exercice 4.}}
	Soient $A$ et $B$ deux ensembles.\\
	Trouver une condition nécessaire et suffisante pour que
	$A \times B = B \times A$
	
\end{document}
