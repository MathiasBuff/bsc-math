% !TeX spellcheck = fr_FR

% TODO: Replace scan images with clean text where possible

\documentclass[a4paper, 10pt]{report}

\usepackage[french]{babel}
\usepackage[T1]{fontenc}

\usepackage{amsmath, amssymb, amsfonts}

\usepackage{hyperref}
\usepackage{geometry}

\usepackage{xcolor}
\usepackage{graphicx}

\usepackage{fancyhdr}
\usepackage{lastpage}

\usepackage{enumitem}

\geometry{
	a4paper,
	left=25mm,
	right=25mm,
	top=35mm,
	bottom=25mm,
	headsep=5mm,
	headheight=20mm,
}

\definecolor{solution}{HTML}{E5E4E2}
\providecommand{\abs}[1]{\lvert#1\rvert}
\providecommand{\norm}[1]{\lVert#1\rVert}

\begin{document}
	
	\renewcommand{\headrule}{%
		\vspace{-4pt}\hrulefill
		\raisebox{-6.8pt}{\ \includegraphics[height=5mm]{../../icon.png}}
		\hrulefill
	}	
	\pagestyle{fancy}
	\fancyhf{}
	
	\fancyhead[L]{\small \slshape Automne 2024}
	\fancyhead[C]{\Large \bfseries Logique et Théorie des Ensembles\\
		Série 05-B}
	\fancyhead[R]{\small Buff Mathias}
	\fancyfoot[L]{
		\small Source files available at:
		\href{https://github.com/MathiasBuff/bsc-math}
		{github.com/MathiasBuff/bsc-math}
	}
	\fancyfoot[R]{
		\small Page \thepage
		\hspace{1pt} /
		\pageref*{LastPage}
	}
	
	\noindent
	\textbf{Exercice 1.} Nier la proposition : "Tous les étudiants
	de la faculté des Sciences qui ont les yeux marron auront 6 à
	tous leurs examens et prendront leur retraite avant 50 ans."
	
	\colorbox{solution}
	{
		\begin{minipage}{0.9\textwidth}
			"Il existe un étudiant de la faculté des Sciences qui a
			les yeux marron et qui n'aura pas 6 à un de ses examens
			ou qui prendra sa retraite après 50 ans."\\
			\textit{\color{blue} (C'est sûrement Philippe)}
		\end{minipage}
	}
		
	\vspace{5mm}
	\noindent
	\textbf{Exercice 2.} Les assertions suivantes sont-elles vraies
	ou fausses ? Donner leur négation.
	
	\begin{enumerate}[label=\arabic*.]
		\item $\exists x \in \mathbb{R}, \forall y \in \mathbb{R},
			x + y > 0$
		%
		\item $\forall x \in \mathbb{R}, \exists y \in \mathbb{R},
			x + y > 0$
		%
		\item $\forall x \in \mathbb{R}, \forall y \in \mathbb{R},
			x + y > 0$
		%
		\item $\exists x \in \mathbb{R}, \exists y \in \mathbb{R},
			x + y > 0$
	\end{enumerate}
	
	\colorbox{solution}
	{
		\begin{minipage}{0.9\textwidth}
			\begin{enumerate}[label=\arabic*.]
				\item FAUSSE.\\ Négation : $\forall x \in \mathbb{R},
					\exists y \in \mathbb{R}, x + y \leq 0$
				%
				\item VRAIE.\\ Négation : $\exists x \in \mathbb{R},
					\forall y \in \mathbb{R}, x + y \leq 0$
				%
				\item FAUSSE.\\ Négation : $\exists x \in \mathbb{R},
					\exists y \in \mathbb{R}, x + y \leq 0$
				%
				\item VRAIE.\\ Négation : $\forall x \in \mathbb{R},
					\forall y \in \mathbb{R}, x + y \leq 0$
			\end{enumerate}
		\end{minipage}
	}
		
	\vspace{5mm}
	\noindent
	\textbf{Exercice 3.} Écrire la négation des assertions suivantes :
	
	\begin{enumerate}[label=\arabic*.]
		\item $\forall x, y \in E, xy = yx$
		%
		\item $\exists x \in E, \forall y \in E, xy = yx$
		%
		\item $\forall a, b \in A, [ab = 0 \implies
			(a = 0\ \text{ou}\ b = 0)]$
		%
		\item $\forall x \in \mathbb{R}, \forall y \in \mathbb{R},
			[x < y \implies f(x) < f(y)]$
		%
		\item $\forall \varepsilon > 0, \exists N \in \mathbb{N},
			[n \geq N \implies \abs{u_n - \ell} < \varepsilon]$
		%
		\item $\exists \ell \in \mathbb{R}, \forall \varepsilon > 0,
			\exists N \in \mathbb{N},
			[n \geq N \implies \abs{u_n - \ell} < \varepsilon]$
	\end{enumerate}
	
	\colorbox{solution}
	{
		\begin{minipage}{0.9\textwidth}
			\begin{enumerate}[label=\arabic*.]
				\item $\exists x, y \in E, xy \neq yx$
				%
				\item $\forall x \in E, \exists y \in E, xy \neq yx$
				%
				\item $\exists a, b \in A, [ab = 0\ \text{et}\
					(a \neq 0\ \text{et}\ b \neq 0)]$
				%
				\item $\exists x \in \mathbb{R}, \exists y \in \mathbb{R},
					[x < y\ \text{et}\ f(x) \geq f(y)]$
				%
				\item $\exists \varepsilon > 0, \forall N \in \mathbb{N},
					[n \geq N\ \text{et}\ \abs{u_n - \ell} \geq \varepsilon]$
				%
				\item $\forall \ell \in \mathbb{R},
					\exists \varepsilon > 0, \forall N \in \mathbb{N},
					[n \geq N\ \text{et}\ \abs{u_n - \ell} \geq \varepsilon]$
			\end{enumerate}
		\end{minipage}
	}
	
	\newpage
	
	\fancyhf{}
	\renewcommand{\headrule}
	{\rule{\textwidth}{0pt}}
	\fancyfoot[R]{
		\small Page \thepage
		\hspace{1pt} /
		\pageref*{LastPage}
	}
	
	\noindent
	\textbf{Exercice 4.} Expliquer verbalement ce que signifient les
	assertions suivantes et écrire leur négation.
	
	\begin{enumerate}[label=\arabic*.]
		\item $\forall n \geq 0, u_n < u_{n+1}$
		(où $(u_n)$ est une suite réelle)
		%
		\item Soit $f: E \to \mathbb{R}$ une fonction :
		\begin{enumerate}[label=(\alph*)]
			\item $\exists C \in \mathbb{R}, \forall x \in E,
				f(x) = C$
			%
			\item $\forall x \in E,
				[f(x) = 0 \implies x = 0]$
			%
			\item $\forall y \in \mathbb{R}, \exists x \in E,
				f(x) = y$
			%
			\item $\forall x \in E, \forall y \in E,
				[f(x) = f(y) \implies x = y]$
			%
			\item $\exists A \in \mathbb{R}, \forall x \in E,
				f(x) \leq A$
		\end{enumerate}
	\end{enumerate}
	
	\colorbox{solution}
	{
		\begin{minipage}{0.9\textwidth}
			\begin{enumerate}[label=\arabic*.]
				\item "$(u_n)$ es strictement croissante"\\
				Négation : $\exists n \geq 0, u_n \geq u_{n+1}$
				%
				\item
				\begin{enumerate}[label=(\alph*)]
					\item "$f$ est constante (en $C$)"\\
					Négation : $\forall C \in \mathbb{R}, \exists x \in E,
						f(x) \neq C$
					%
					\item "$f(x) = 0$ uniquement pour $x = 0$"\\
					Négation : $\exists x \in E,
						[f(x) = 0\ \text{et}\ x \neq 0]$
					%
					\item "$f$ est surjective"\\
					Négation : $\exists y \in \mathbb{R}, \forall x \in E,
						f(x) \neq y$
					%
					\item "$f$ est injective"\\
					Négation : $\exists x \in E, \exists y \in E,
						[f(x) = f(y)\ \text{et}\ x \neq y]$
					%
					\item "$f$ est majorée (par $A$)"\\
					Négation : $\forall A \in \mathbb{R}, \exists x \in E,
						f(x) > A$
				\end{enumerate}
			\end{enumerate}
		\end{minipage}
	}
	
	\vspace{5mm}
	\noindent
	{\color{red}\textbf{Exercice 5.}}
	Soit $E$ un ensemble et $P(x)$ des prédicats indexés par $x \in E$.
	Écrire l'assertion $\exists! x \in E, P(x)$ à l'aide des
	quantificateurs $\exists$ et $\forall$. Puis écrire la négation de
	cette assertion.
	
	\colorbox{solution}
	{
		\begin{minipage}{0.9\textwidth}
			$\exists! x \in E, P(x)$ : \textit{Il existe un $x$ pour lequel
			$P(x)$ sont vrais, et ce $x$ est unique}\\ 
			$\iff \exists x \in E, P(x)\ \text{et}\
				[\forall y \in E, P(y) \implies x = y]$\\
				
			Négation : \textit{Soit il n'existe aucun $x$ pour lequel
			$P(x)$ sont vrais, soit $x$ n'est pas unique}\\
			$\iff \forall x \in E, \lnot P(x)\ \text{ou}\
				[\exists y \in E, P(y)\ \text{et}\ x = y]$
		\end{minipage}
	}
	
\end{document}
