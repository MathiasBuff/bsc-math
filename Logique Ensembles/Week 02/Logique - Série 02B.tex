% !TeX spellcheck = fr_FR

% TODO: Replace scan images with clean text where possible

\documentclass[a4paper, 10pt]{report}

\usepackage[french]{babel}
\usepackage[T1]{fontenc}

\usepackage{amsmath, amssymb, amsfonts}

\usepackage{hyperref}
\usepackage{geometry}

\usepackage{xcolor}
\usepackage{graphicx}

\usepackage{fancyhdr}
\usepackage{lastpage}

\usepackage{enumitem}

\geometry{
	a4paper,
	left=25mm,
	right=25mm,
	top=35mm,
	bottom=25mm,
	headsep=5mm,
	headheight=20mm,
}

\definecolor{solution}{HTML}{E5E4E2}
\providecommand{\abs}[1]{\lvert#1\rvert}
\providecommand{\norm}[1]{\lVert#1\rVert}

\begin{document}
	
	\renewcommand{\headrule}{%
		\vspace{-4pt}\hrulefill
		\raisebox{-6.8pt}{\ \includegraphics[height=5mm]{../../icon.png}}
		\hrulefill
	}	
	\pagestyle{fancy}
	\fancyhf{}
	
	\fancyhead[L]{\small \slshape Automne 2024}
	\fancyhead[C]{\Large \bfseries Logique et Théorie des Ensembles\\
		Série 02-B}
	\fancyhead[R]{\small Buff Mathias}
	\fancyfoot[L]{
		\small Source files available at:
		\href{https://github.com/MathiasBuff/bsc-math}
		{github.com/MathiasBuff/bsc-math}
	}
	\fancyfoot[R]{
		\small Page \thepage
		\hspace{1pt} /
		\pageref*{LastPage}
	}
	
	\noindent
	\textbf{Exercice 1.} En partant de l'ensemble vide, construire à
	l'aide des axiomes des ensembles à $10$ et $2^{10}$ éléments.
	Pensez-vous pouvoir construire des ensembles de taille finie
	quelconque ?
	
	\vspace{5mm}
	\noindent
	\textbf{Exercice 2.} (Différence symétrique). L'opération
	$\bigtriangleup$ est définie sur les ensembles $A, B \subset E$\\
	par $A \bigtriangleup B =
		(A \cap (E \setminus B)) \cup (B \cap (E \setminus A))$.
	
	\begin{enumerate}[label=\arabic*.]
		\item Montrer que : $A \bigtriangleup B =
			(A \cup B) \setminus (A \cap B)$.
		%
		\item Vérifier que : $A \bigtriangleup B = \emptyset
			\iff (A = B)$.
	\end{enumerate}
	
	\vspace{5mm}
	\noindent
	\textbf{Exercice 3.} Soient $E, F$, et $G$ trois ensembles.
	
	\begin{enumerate}[label=\arabic*.]
		\item Montrer que $E \cup (F \cap G)
			= (E \cup F) \cap (E \cup G)$.
		%
		\item Montrer que $E \setminus (F \cup G)
			= (E \setminus F) \cap (E \setminus G)$.
		%
		\item Montrer que $E \setminus (F \cap G)
		= (E \setminus F) \cup (E \setminus G)$.
	\end{enumerate}
	
	\vspace{5mm}
	\noindent
	\textbf{Exercice 4.} Montrer que $E = F$ si et seulement si
	$\mathcal{P}(E) = \mathcal{P}(F)$.
	
	\vspace{5mm}
	\noindent
	\textbf{Exercice 5.} Expliciter les éléments de l'ensemble
	$\mathcal{P}(\mathcal{P}(\{0\}))$.
	
	\vspace{5mm}
	\noindent
	{\color{red}\textbf{Exercice 6.}}
	Montrer qu'il n'existe pas d'ensemble $F$ de tous les ensembles
	(on pourra raisonner par l'absurde et considérer la partie de $F$
	composée des ensembles ne s'appartenant pas).	
\end{document}
