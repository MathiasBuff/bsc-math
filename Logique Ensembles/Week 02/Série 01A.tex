% !TeX spellcheck = fr_FR

% TODO: Replace scan images with clean text where possible

\documentclass[a4paper, 10pt]{report}

\usepackage[french]{babel}
\usepackage[T1]{fontenc}

\usepackage{amsmath, amssymb, amsfonts}

\usepackage{hyperref}
\usepackage{geometry}

\usepackage{xcolor}
\usepackage{graphicx}

\usepackage{fancyhdr}
\usepackage{lastpage}

\usepackage{enumitem}

\geometry{
	a4paper,
	left=25mm,
	right=25mm,
	top=35mm,
	bottom=25mm,
	headsep=5mm,
	headheight=20mm,
}

\definecolor{solution}{HTML}{E5E4E2}
\providecommand{\abs}[1]{\lvert#1\rvert}
\providecommand{\norm}[1]{\lVert#1\rVert}

\begin{document}
	
	\renewcommand{\headrule}{%
		\vspace{-4pt}\hrulefill
		\raisebox{-6.8pt}{\ \includegraphics[height=5mm]{../../icon.png}}
		\hrulefill
	}	
	\pagestyle{fancy}
	\fancyhf{}
	
	\fancyhead[L]{\small \slshape Automne 2024}
	\fancyhead[C]{\Large \bfseries Logique et Théorie des Ensembles\\
		Série 01-A}
	\fancyhead[R]{\small Buff Mathias}
	\fancyfoot[L]{
		\small Source files available at:
		\href{https://github.com/MathiasBuff/bsc-math}
		{github.com/MathiasBuff/bsc-math}
	}
	\fancyfoot[R]{
		\small Page \thepage
		\hspace{1pt} /
		\pageref*{LastPage}
	}
	
	\noindent
	\textbf{Exercice 1.} Soient $E, F$, et $G$ trois ensembles.\\
	\indent
	Montrer que si $E \subset G$ et $F \subset G$,
	alors $E \cup F \subset G$.
	
	
	\vspace{5mm}
	\noindent
	\textbf{Exercice 2.} Soient $A, B$, et $C$ trois ensembles.
	
	\begin{enumerate}[label=\arabic*.]
		\item Montrer que : $(A = B) \iff (A \cup B \subset A \cap B)$.
		%
		\item Montrer que :
		 $(A \subset B) \iff (\mathcal{P}(A) \subset \mathcal{P}(B))$.
		%
		\item Montrer que : $(A \cup B \subset A \cup C
			\ \text{et}\ A \cap B \subset A \cap C)
			\implies (B \subset C)$
	\end{enumerate}
	
	\vspace{5mm}
	\noindent
	\textbf{Exercice 3.} Dites si les propositions suivantes sont
	VRAIES ou FAUSSES :
	
	\begin{enumerate}[label=\arabic*.]
		\item $\mathbb{Q} \in \mathbb{R}$
		%
		\item $\mathbb{Q} \subset \mathbb{R}$
		%
		\item $\{\emptyset\} \in \mathcal{P}(\mathbb{N})$
		%
		\item $\{\emptyset\} \subset \mathcal{P}(\mathbb{N})$
		%
		\item $\emptyset \subset \mathcal{P}(\mathbb{N})$
		%
		\item $\{\{1\}\} \in \mathcal{P}(\{1, 2, 3\})$
		%
		\item $\{\{1\}\} \subset \mathcal{P}(\{1, 2, 3\})$
	\end{enumerate}
	
	\vspace{5mm}
	\noindent
	\textbf{Exercice 4.} Considérons les sous-ensembles de
	$\mathbb{N}$ suivants :
	
	\[
		A = \{1, 2, 3, 4, 5, 6, 7\} \quad
		B = \{1, 2, 5, 7\} \quad
		C = \{2, 3, 4, 6\} \quad
		D = \{3, 6\}
	\]
	
	\begin{enumerate}[label=\arabic*.]
		\item Déterminer $B \cap D$ et $C \cap D$.
		%
		\item Déterminer $B \cup D$ et $C \cup D$. L'une de ces deux
		unions est-elle disjointe ?
		%
		\item Déterminer les complémentaires dans $A$ de $B, C$, et $D$.
	\end{enumerate}
	
	\vspace{5mm}
	\noindent
	{\color{red}\textbf{Exercice 5.}}
	On se donne  $A, B$ des parties de $E$.
	
	\begin{enumerate}[label=\arabic*.]
		\item Donner une condition nécessaire et suffisante pour que
		l'équation $A \cap X = B$, où $X = \mathcal{P}(E)$ est
		l'inconnue, admette au moins une solution, et résoudre alors
		l'équation.
		%
		\item Même question pour $A \cup X = B$
		%
		\item Même question pour $A \bigtriangleup X = B$, où
		$\bigtriangleup$ désigne la différence symétrique
		$A \bigtriangleup B = (A \setminus B) \cup (B \setminus A)$.
	\end{enumerate}
\end{document}
