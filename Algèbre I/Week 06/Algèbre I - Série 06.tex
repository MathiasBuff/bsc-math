% !TeX spellcheck = fr_FR

% TODO: Replace scan images with clean text where possible

\documentclass[a4paper, 10pt]{report}

\usepackage[french]{babel}
\usepackage[T1]{fontenc}

\usepackage{amsmath, amssymb, amsfonts}

\usepackage{hyperref}
\usepackage{geometry}

\usepackage{xcolor}
\usepackage{graphicx}

\usepackage{fancyhdr}
\usepackage{lastpage}

\usepackage{enumitem}

\geometry{
	a4paper,
	left=25mm,
	right=25mm,
	top=35mm,
	bottom=25mm,
	headsep=5mm,
	headheight=20mm,
}

\definecolor{solution}{HTML}{E5E4E2}
\providecommand{\abs}[1]{\lvert#1\rvert}
\providecommand{\norm}[1]{\lVert#1\rVert}
\DeclareMathOperator{\card}{card}

\makeatletter
\renewcommand*\env@matrix[1][*\c@MaxMatrixCols c]{%
	\hskip -\arraycolsep
	\let\@ifnextchar\new@ifnextchar
	\array{#1}}
\makeatother

\begin{document}
	
	\renewcommand{\headrule}{%
		\vspace{-4pt}\hrulefill
		\raisebox{-6.8pt}{\ \includegraphics[height=5mm]{../../icon.png}}
		\hrulefill
	}	
	\pagestyle{fancy}
	\fancyhf{}
	
	\fancyhead[L]{\small \slshape Automne 2024}
	\fancyhead[C]{\Large \bfseries Algèbre I - Série 06}
	\fancyhead[R]{\small Buff Mathias}
	\fancyfoot[L]{
		\small Source files available at:
		\href{https://github.com/MathiasBuff/bsc-math}
		{github.com/MathiasBuff/bsc-math}
	}
	\fancyfoot[R]{
		\small Page \thepage
		\hspace{1pt} /
		\pageref*{LastPage}
	}
	

	\noindent
	\textbf{Exercice 1.} (Exemples d'applications linéaires)\\
	Prouver si ces applications sont linéaires ou non. Si oui, décrire
	le noyau et l'image, et en donner des bases. Discuter de
	l'injectivité et du rang de ces fonctions.
	\begin{enumerate}[label=\arabic*.]
		\item (Sur $\mathbb{R}$)
		\[f_1: \begin{aligned}
			&\mathbb{R}^3 &\to& \quad \mathbb{R}^3\\
			&(x, y, z) &\mapsto&\quad (x+2y-3z, 7x+6y-13z, -2x+4y-2z)
		\end{aligned}\]
		%
		\item (Sur $\mathbb{R}$)
		\[f_2: \begin{aligned}
			&\mathbb{C} &\to& \quad \mathbb{C}\\
			&z &\mapsto&\quad -\tfrac{1}{2}\bar{z}
		\end{aligned}\]
		%
		\item (Sur $\mathbb{C}$)
		\[f_3: \begin{aligned}
			&\mathbb{C} &\to& \quad \mathbb{C}\\
			&z &\mapsto&\quad -\tfrac{1}{2}\bar{z}
		\end{aligned}\]
		%
		\item (Sur $\mathbb{R}$)
		\[f_4: \begin{aligned}
			&\mathbb{R}_n[x] &\to& \quad \mathbb{R}_n[x]\\
			&p(x) &\mapsto&\quad p(x+1)
		\end{aligned}\]
		%
		\item (Sur $\mathbb{R}$)
		\[f_5: \begin{aligned}
				&\mathbb{R}_n[x] &\to& \quad \mathbb{R}_n[x]\\
			&p(x) &\mapsto&\quad p(x)+1
		\end{aligned}\]
		%
		\item (Sur $\mathbb{R}$)
		\[f_6: \begin{aligned}
			&\mathbb{R}_2[x] &\to& \quad \mathbb{R}^2\\
			&p(x) &\mapsto&\quad (p(0), p'(0))
		\end{aligned}\]
		%
		\item (Sur $\mathbb{R}$)
		\[f_7: \begin{aligned}
			&\mathbb{R}_2[x] &\to& \quad \mathbb{R}^3\\
			&p(x) &\mapsto&\quad (p(0), p(1), p(2))
		\end{aligned}\]
	\end{enumerate}
		
	\vspace{5mm}
	\noindent
	\textbf{Exercice 2.} (Une application linéaire particulière)
	Soit $E$ un espace vectoriel de dimension finie $n$ sur un corps
	$\mathbb{K}$, et $f: E \to E$ une application linéaire. Montrer que
	les assertions suivantes sont équivalentes :
	\begin{enumerate}[label=\arabic*.]
		\item Ker($f$) = Im($f$)
		%
		\item $f \circ f = 0$ et $2 \cdot$rang($f$) = $n$
	\end{enumerate}
	
	\vspace{5mm}
	\noindent
	\textbf{Exercice 3.} (Somme d'applications linéaires)\\
	Soient $f, g: V \to W$ des applications linéaires, où $V$ et $W$
	sont des espaces vectoriels sur un corps $\mathbb{K}$.
	\begin{enumerate}[label=\arabic*.]
		\item Montrer que la somme
		\[
			f+g: \begin{aligned}
				&V &\to& \quad W\\
				&x &\mapsto&\quad f(x)+g(x)
			\end{aligned}
		\]
		est une application linéaire.\\
		\textit{Remarque : voir la Proposition 3.1.1.}
		%
		\item Montrer que Ker$(f)\ \cap$ Ker$(g) \subset$ Ker$(f+g)$.\\
		L'inclusion dans l'autre sens est-elle vraie ? Si non, donner
		un contre-exemple.
		%
		\item Montrer que Im$(f+g) \subset$ Im$(f)\ +$ Im$(g)$.\\
		L'inclusion dans l'autre sens est-elle vraie ? Si non, donner
		un contre-exemple.
	\end{enumerate}
	
	\newpage
	
	\vspace{5mm}
	\noindent
	\textbf{Exercice 4.} (Base et application linéaire)\\
	Soient $E$ un espace vectoriel de dimension finie $n$ sur un corps
	$\mathbb{K}$, et $f: E \to E$ un endomorphisme tel que $f^n = 0$
	et $f^{n-1} \neq 0$. Soit $x$ tel que $f^{n-1}(x) \neq 0$. Montrer
	que la famille $\{x, f(X), f^2(x), \dotsc, f^{n-1}(x)\}$ est une
	base de $E$.
	
	\vspace{5mm}
	\noindent
	\textbf{Exercice 5.} (Projection linéaire)\\
	Soit $V$ un espace vectoriel sur un corps $\mathbb{K}$. Une
	application linéaire $p: V <to V$ est appelée une \textit{projection}
	si $p \circ p = p$ Il y a deux exemples de projections
	\textit{triviales} : l'application identité et l'application nulle.
	\begin{enumerate}[label=\arabic*.]
		\item Pour $V = \mathbb{R}^2$ et $\mathbb{K} = \mathbb{R}$,
		donner un exemple de projection qui n'est pas triviale.
		%
		\item Soit $p$ une projection. Montrer que Ker$(p)\ \cap$ Im$(p)
			= \{0\}$.
		%
		\item Soit $p$ une projection. Montrer que Ker$(p)\ +$ Im$(p)
			= V$.
		%
		\item Conclure que $V = $Ker$(p) \oplus\ $Im$(p)$. Quel est le
		lien entre l'application $p$ et les les applications linéaires
		P\textsubscript{Ker($p$)} et P\textsubscript{Im($p$)} introduites dans l'exemple 3.3.1(4) ?
		%
		\item Conclure que toute projection comme définie ci-dessus est de
		type projection sur un sous-espace $U$ parallèlement à un
		sous-espace $W$ où $W$ est complémentaire à $U$. 
	\end{enumerate}
	
	\fancyhf{}
	\renewcommand{\headrule}
	{\rule{\textwidth}{0pt}}
	\fancyfoot[R]{
		\small Page \thepage
		\hspace{1pt} /
		\pageref*{LastPage}
	}
	
%	
%	\colorbox{solution}
%	{
%		\begin{minipage}{0.9\textwidth}
%			s
%		\end{minipage}
%	}
%	
%	\colorbox{solution}
%	{
%		\begin{minipage}{0.9\textwidth}
%			\begin{enumerate}[label=(\alph*)]
%				\item a
%			\end{enumerate}
%		\end{minipage}
%	}
	
\end{document}
