% !TeX spellcheck = fr_FR

% TODO: Replace scan images with clean text where possible

\documentclass[a4paper, 10pt]{report}

\usepackage[french]{babel}
\usepackage[T1]{fontenc}

\usepackage{amsmath, amssymb, amsfonts}

\usepackage{hyperref}
\usepackage{geometry}

\usepackage{xcolor}
\usepackage{graphicx}

\usepackage{fancyhdr}
\usepackage{lastpage}

\usepackage{enumitem}

\geometry{
	a4paper,
	left=25mm,
	right=25mm,
	top=35mm,
	bottom=25mm,
	headsep=5mm,
	headheight=20mm,
}

\definecolor{solution}{HTML}{E5E4E2}
\providecommand{\abs}[1]{\lvert#1\rvert}
\providecommand{\norm}[1]{\lVert#1\rVert}

\begin{document}
	
	\renewcommand{\headrule}{%
		\vspace{-4pt}\hrulefill
		\raisebox{-6.8pt}{\ \includegraphics[height=5mm]{../../icon.png}}
		\hrulefill
	}	
	\pagestyle{fancy}
	\fancyhf{}
	
	\fancyhead[L]{\small \slshape Automne 2024}
	\fancyhead[C]{\Large \bfseries Algèbre I - Série 02}
	\fancyhead[R]{\small Buff Mathias}
	\fancyfoot[L]{
		\small Source files available at:
		\href{https://github.com/MathiasBuff/bsc-math}
		{github.com/MathiasBuff/bsc-math}
	}
	\fancyfoot[R]{
		\small Page \thepage
		\hspace{1pt} /
		\pageref*{LastPage}
	}
	
	\noindent
	\textbf{Exercice 1.} (Nombres complexes)
	
	\begin{enumerate}[label=\arabic*.]
		\item Écrire en forme algébrique $z = a + ib$ avec
		$a, b \in \mathbb{R}$ et $i^2 = -1$ les nombres complexes
		suivants :
		%
		\[
			\frac{1}{i(3 + 2i)^2}, \quad
			\frac{(\sqrt{3} + \sqrt{2}i)^3}{\sqrt{2} - \sqrt{3}i}, \quad
			\frac{1}{\sqrt{3} - i}.
		\]
		%
		\includegraphics{ex01-1.jpg}
		%
		\item Calculer le module des nombres complexes suivants :
		%
		\[
			-3i, \quad
			\sqrt{3} + i, \quad
			3i(2 + i).
		\]
		%
		\includegraphics{ex01-2.jpg}
		%
		\item Résoudre les équations suivantes d'inconnue
		$z \in \mathbb{C}$ avec la formule habituelle pour résoudre
		les équations générales du second degré. Vérifier les
		solutions trouvées.
		%
		\begin{itemize}
			\item $z^2 + 2z + 3 = 0$,
			\item $z^2 + 2iz - 3 = 0$.
		\end{itemize}
		%
		\includegraphics{ex01-3.jpg}
		%
		\newpage
		
		\item Résoudre l'équation suivante d'inconnue
		$z \in \mathbb{C}$ en remplaçant $z = a + ib$
		%
		\[
			z + 3i + Re(z)(i + (Im(z))^2) = 0
		\]
		%
		\includegraphics{ex01-4.jpg}
		%
		\item Calculer le conjugué en forme algébrique $z = a + ib$
		des nombres complexes suivants :
		%
		\[
			\frac{5 + 2i}{1 - i}, \quad
			\frac{\sqrt{2} - i}{2 + i}, \quad
			\frac{2 - i}{i}.
		\]
		%
		\includegraphics{ex01-5.jpg}
		%
		\item Montrer la proposition suivante :
		$\forall z \in \mathbb{C}$ :
			%
		\begin{itemize}
			\item si $z \neq 0 : \abs{z^{-1}} = \abs{z}^{-1}$ ;
			\item $z \in \mathbb{R} \iff \bar{z} = z$ ;
			\item $z \in i\mathbb{R} \iff \bar{z} = -z$.
		\end{itemize}
		%
		\includegraphics{ex01-6.jpg}
		%
	\end{enumerate}
	
	\fancyhf{}
	\renewcommand{\headrule}
	{\rule{\textwidth}{0pt}}
	\fancyfoot[R]{
		\small Page \thepage
		\hspace{1pt} /
		\pageref*{LastPage}
	}
	
	\newpage
	
	\noindent
	\textbf{Exercice 2.} (Coordonnées polaires)
	
	\begin{enumerate}[label=(\alph*)]
		\item Montrer que
		\[
			z_1z_2 = r_1r_2\big(\cos(\alpha_1 + \alpha_2)
				+ i\sin(\alpha_1 + \alpha_2)\big)
		\]
		où $z = r(\cos\alpha + i \sin\alpha)$ est l'écriture du
		nombre complexe $z$ en coordonnées polaires.
		%
		\item Déduire une formule analogue pour $\frac{z_1}{z_2}$
		lorsque $z_2 \neq 0$.
		%
		\item Représenter graphiquement les solutions d'équation
		$z^3 = 1$ (racines cubiques de l'unité).
	\end{enumerate}
	%
	\includegraphics{ex02.jpg}
	
	\newpage
	
	\noindent
	\textbf{Exercice 3.} (Propriétés d'espace vectoriel)\\
	Soit $V$ un espace vectoriel sur un corps $K$. Démontrer que les
	propriétés suivantes sont satisfaites
	
	\begin{enumerate}[label=\arabic*.]
		\item \begin{enumerate}[label=(\alph*)]
			\item L'élément neutre pour l'addition est unique;
			%
			\item $\forall v \in V$, l'opposé $(-v)$ de $v$ est unique;
		\end{enumerate}
		%
		\item \begin{enumerate}[label=(\alph*)]
			\item $0_K \cdot v = 0_V \quad \forall v \in V$;
			%
			\item $\lambda \cdot 0_V = 0_V \quad \forall \lambda \in K$;
		\end{enumerate}
		%
		\item L'opposé de $v$, noté $-v$, satisfait $-v = (-1) \cdot v$
		avec $v \in V$ et $-1 \in K$;
		%
		\item $\lambda \cdot v = 0 \iff
			\lambda = 0_K \ \text{ou}\ v = 0_V$.
	\end{enumerate}
	
	\noindent
	\textit{NB : Cet exercice figure en tant que Proposition 1.1.1 du
		polycopié, mais vous pouvez essayer de trouver vous-mêmes
		les preuves avant de regarder le polycopié.}
	
	\includegraphics{ex03.jpg}
	
	\newpage
	
	\noindent
	\textbf{Exercice 4.} (Un exemple d'espace vectoriel)\\
	Définir les opérations $+$ et $\cdot$ sur $\mathbb{C}^n$ et
	vérifier que cela munit $\mathbb{C}^n$ d'une structure d'espace
	vectoriel sur $\mathbb{C}$. Détailler l'associativité de la
	multiplication par les scalaires.
	
	\includegraphics{ex04.jpg}
	
	\newpage
	
	\noindent
	\textbf{Exercice 5.} (Est ce un sous-espace vectoriel ?
	Trouver une famille génératrice.)
	
	\begin{enumerate}[label=\arabic*.]
		\item Les sous-ensembles suivants de $\mathbb{R}^2$ sont-ils
		des sous-espaces vectoriels sur $\mathbb{R}$ ? Si oui, trouver
		une famille génératrice.
		\begin{enumerate}[label=(\alph*)]
			\item $E = \{(x, y) \in \mathbb{R}^2 \mid x - y = 0\}$
			\item $E = \{(x, y) \in \mathbb{R}^2 \mid xy - x - y = 0\}$
		\end{enumerate}
		%
		\colorbox{solution}
		{
			\begin{minipage}{0.9\textwidth}
				\begin{enumerate}[label=(\alph*)]
					\item \underline{Oui} : Il s'agit de la droite dans
					$\mathbb{R}^2$ d'équation $y = x$.\\
					$\left\{\left(\begin{smallmatrix}
						1\\
						1
					\end{smallmatrix}\right)\right\}$
					est une famille génératrice de cette droite.
					%
					\item \underline{Non} : $(2, 2) \in E$, mais
					$2 \cdot (2, 2) = (4, 4) \notin E$, donc $E$ n'est pas
					clos pour la multiplication par un scalaire (il est
					également facile de vérifier qu'il n'est pas clos
					pour l'addition), par conséquent $E$ n'est pas un
					sous-espace vectoriel de $\mathbb{R}^2$.
				\end{enumerate}
			\end{minipage}
		}
		%
		\item Les sous-ensembles suivants de $\mathbb{C}^3$ sont-ils
		des sous-espaces vectoriels sur $\mathbb{C}$ ? Si oui, trouver
		une famille génératrice.
		\begin{enumerate}[label=(\alph*)]
			\item $E = \{(x, y, z) \in \mathbb{C}^3 \mid x + 2y + 3z = 1\}$
			\item $E = \{(x, y, z) \in \mathbb{C}^3 \mid y = 0\}$
			\item $E = \{(x, 2x, 3x) \mid x \in \mathbb{R}\}$
			\item $E = \{(x, 2x, 3x) \mid x \in \mathbb{C}\}$
			\item $E = \{(x, y, z) \in \mathbb{C}^3 \mid y^2 -x^3 = 0\}$
		\end{enumerate}
		%
		\colorbox{solution}
		{
			\begin{minipage}{0.9\textwidth}
				\begin{enumerate}[label=(\alph*)]
					\item \underline{Non} : 
						$0_{\mathbb{C}^3} = (0, 0, 0)$
						n'appartient pas à $E$
					%
					\item \underline{Oui} : Il s'agit du "plan" formé
						par les "axes" $x$ et $z$.\\
						$\left\{\left(\begin{smallmatrix}
							1\\ 0 \\ 0
						\end{smallmatrix}\right),
						\left(\begin{smallmatrix}
							i\\ 0 \\ 0
						\end{smallmatrix}\right),
						\left(\begin{smallmatrix}
							0\\ 0 \\ 1
						\end{smallmatrix}\right),
						\left(\begin{smallmatrix}
							0\\ 0 \\ i
						\end{smallmatrix}\right)\right\}$
						est génératrice de ce sous-espace vectoriel.
					%
					\item \underline{Non} : Si $v \in E$, alors
						$i \cdot v \notin E$, donc $E$ n'est pas clos
						pour la multiplication par un scalaire dans
						$\mathbb{C}$.
					%
					\item \underline{Oui} : Il s'agit du plan formé par
						les points $(0, 0, 0), (1, 2, 3), (i, 2i, 3i)$.\\
						$\left\{\left(\begin{smallmatrix}
							1\\ 2 \\ 3
						\end{smallmatrix}\right),
						\left(\begin{smallmatrix}
							i\\ 2i \\ 3i
						\end{smallmatrix}\right)\right\}$
						est génératrice de ce plan.
					%
					\item \underline{Non} : $(1, 1, 1) \in E$, mais
						$(1, 1, 1) + (1, 1, 1) = (2, 2, 2) \notin E$,
						donc $E$ n'est pas clos pour l'addition.
						(Il n'est également pas clos pour la
						multiplication par un scalaire.)
				\end{enumerate}
			\end{minipage}
		}
	\end{enumerate}

	
	\newpage
	
	\noindent
	\textbf{Exercice 6.} L’objectif de cet exercice est de
	vérifier que toutes les propriétés demandées dans la définition
	de sous-espace vectoriel sont nécessaires.
	
	\begin{enumerate}[label=\arabic*.]
		\item Donner un exemple de sous-ensemble non vide $U$ de
		$\mathbb{R}^2$ qui vérifie
		\[\begin{split}
			&\forall u \in U, \forall v \in U, \quad u + v \in U,\\
			&\forall u \in U, \quad -u \in U
		\end{split}\]
		mais qui ne soit pas un sous-espace vectoriel de $\mathbb{R}^2$.
		
		\colorbox{solution}
		{
			\begin{minipage}{0.9\textwidth}
				Le sous-ensemble $U = \mathbb{Z}^2$ satisfait les
				conditions données :\\
				Soient $u = (a, b), v = (c, d),
					\quad a, b, c, d \in \mathbb{Z}$, alors
				\[\begin{split}
					&(a, b) + (c, d) = (a+c, b+d) \in U\\
					&-u = (-a, -b) \in U
				\end{split}\]
				En revanche, $[\forall u \in U,
					\forall \lambda \in \mathbb{R}, \quad
					\lambda \cdot u \in U]$ n'est pas vérifiée,
				il suffit de prendre $\lambda = \frac{1}{a}$
			\end{minipage}
		}
		%
		\item Donner un exemple de sous-ensemble non vide $U$ de
		$\mathbb{R}^2$ qui vérifie
		\[
			\forall u \in U, \forall \lambda \in \mathbb{R}, \quad
				\lambda \cdot u \in U
		\]
		mais qui ne soit pas un sous-espace vectoriel de $\mathbb{R}^2$.
		
		\colorbox{solution}
		{
			\begin{minipage}{0.9\textwidth}
				Le sous-ensemble $U = \left\{(x, \abs{x}) \mid 
					x \in \mathbb{R} \right\}$ satisfait les
				conditions données :\\
				Soient $u \in U, \lambda \in \mathbb{R}$, alors
				\[
					\forall \lambda \in \mathbb{R}, \quad
						\lambda \cdot u = \lambda \cdot (x, \abs{x})
						= (\lambda \cdot x, \lambda \cdot \abs{x})
						= (\lambda x, \abs{\lambda x}) \in U
				\]
				En revanche, $[\forall u \in U, \quad -u \in U]$ n'est
				pas vérifiée, car $(-x, -\abs{x}) \notin U$.
			\end{minipage}
		}
		%
		\item Donner un exemple de sous-ensemble non vide $U$ de
		$\mathbb{R}^2$ qui vérifie
		\[\begin{split}
			&\forall u \in U, \forall v \in U, \quad u + v \in U,\\
			&\forall u \in U, \forall \lambda > 0, \quad
				\lambda \cdot u \in U,
		\end{split}\]
		mais qui ne soit pas un sous-espace vectoriel de $\mathbb{R}^2$.
		
		\colorbox{solution}
		{
			\begin{minipage}{0.9\textwidth}
				Le sous-ensemble $U = \left\{(x, y) \mid 
					x, y \in \mathbb{R}_+ \right\}$ satisfait les
				conditions données :\\
					Soient $u = (x_1, y_1), v = (x_2, y_2),
				\quad x_1, y_1, x_2, y_2 \in \mathbb{R}_+$, alors
				\[\begin{split}
					&u + v = (x_1, y_1) + (x_2, y_2)
						= (x_1 + x_2, y_1 + y_2) \in U\\
					&\forall \lambda > 0, \quad \lambda \cdot u
						= \lambda \cdot (x_1, y_1)
						= (\lambda x_1, \lambda y_1) \in U
				\end{split}\]
				En revanche, $0_{\mathbb{R}^2} = (0, 0)$ pas un
				élément de $U$.
			\end{minipage}
		}
	\end{enumerate}
		
	\newpage
	
	\noindent
	L'exercice suivant est \textbf{facultatif}. Il ne sera pas
	corrigé lors de la séance d'exercices, mais sa solution sera
	postée sur Moodle.
	
	\vspace{10mm}
	\noindent
	\textbf{Exercice 7.} (Un corps à 3 éléments)\\
	L’objectif de cet exercice est de construire un corps fini.
	
	\noindent
	Rappelons que le reste de la division d’un entier $n$ par $3$
	est l’unique entier $r$ entre $0$ et $3 - 1$ (compris) tel
	que $n - r$ soit divisible par $3$. Par exemple, le reste de
	$5$ est $2$, le reste de $39$ est $0$, le reste de $-2$ est $1$,
	et le reste de votre numéro d’étudiant est votre groupe pour
	les séances d’exercices.\\
	
	\noindent
	Considérons $K = \{0, 1, 2\}$ et définissons les opérations
	\[
	\begin{split}
		&a \oplus b = \text{le reste de la division de }
			a + b \text{ par } 3\\
		&a \otimes b = \text{le reste de la division de }
			a \cdot b \text{ par } 3
	\end{split}
	\]
	Remplir les tables d'addition et de multiplication suivantes.
	
	\begin{center}
	\begin{tabular}{l|l|l|l|}
		$\oplus$ & 0 & 1 & 2 \\
		\hline
		0 & & & \\
		\hline
		1 & & & \\
		\hline
		2 & & & \\
		\hline
	\end{tabular}
	\qquad
	\begin{tabular}{l|l|l|l|}
		$\otimes$ & 0 & 1 & 2 \\
		\hline
		0 & & & \\
		\hline
		1 & & & \\
		\hline
		2 & & & \\
		\hline
	\end{tabular}
	\end{center}
	
	\noindent
	Montrer l'existence des neutres additif et multiplicatif. Montrer
	que tout élément admet un inverse pour $\oplus$, et tout élément
	non-nul admet un inverse pour $\otimes$.\\
	Vérifier que l'addition	est commutative.
	
	\colorbox{solution}
	{
		\begin{minipage}{0.9\textwidth}
			\vspace{6pt}
			\begin{center}
				\begin{tabular}{l|l|l|l|}
					$\oplus$ & 0 & 1 & 2 \\
					\hline
					0 & 0 & 1 & 2 \\
					\hline
					1 & 1 & 2 & 0 \\
					\hline
					2 & 2 & 0 & 1 \\
					\hline
				\end{tabular}
				\qquad
				\begin{tabular}{l|l|l|l|}
					$\otimes$ & 0 & 1 & 2 \\
					\hline
					0 & 0 & 0 & 0 \\
					\hline
					1 & 0 & 1 & 2 \\
					\hline
					2 & 0 & 2 & 1 \\
					\hline
				\end{tabular}
			\end{center}
			
			\begin{itemize}
				\item $0$ est l'élément neutre pour $\oplus$, car
				$\forall x \in \{0, 1, 2\}, 0 \oplus x = x \oplus 0 = x$.
				%
				\item $1$ est l'élément neutre pour $\otimes$, car
				$\forall x \in \{0, 1, 2\}, 1 \otimes x = x \otimes 1 = x$.
				%
				\item On observe que :
				\begin{itemize}
					\item $0 \oplus 0 = 0$, donc $0$ est l'inverse
					pour $\oplus$ de $0$
					%
					\item $1 \oplus 2 = 0$, donc $2$ est l'inverse
					pour $\oplus$ de $1$
					%
					\item $2 \oplus 1 = 0$, donc $1$ est l'inverse
					pour $\oplus$ de $2$
					%
				\end{itemize}
				%
				\item On observe que :
				\begin{itemize}
					\item $1 \otimes 1 = 1$, donc $1$ est l'inverse
					pour $\otimes$ de $1$
					%
					\item $2 \otimes 2 = 1$, donc $2$ est l'inverse
					pour $\otimes$ de $2$
					%
				\end{itemize}
				%
				\item Enfin, la table d'addition étant symétrique selon
				sa diagonale, on peut en déduire que l'addition est
				commutative (la même observation peut être faite
				pour la multiplication).
			\end{itemize}
		\end{minipage}
	}
\end{document}
